\documentclass[14pt]{extarticle}
\usepackage[a4paper,margin=25mm]{geometry}
\usepackage{xcolor}
\usepackage{amsmath}      
\usepackage{amssymb}        
\usepackage{amsthm}         
\theoremstyle{definition}            
\newtheorem{definition}{Определение}[section]
\newtheorem{example}{Пример}[section]
\DeclareMathOperator{\avg}{avg}
\DeclareMathOperator{\shape}{shape}
\usepackage{polyglossia}
\setmainlanguage{russian}     
\setotherlanguage{english}
\usepackage{fontspec}
\usepackage{caption}
\setmainfont{Times New Roman}[Ligatures=TeX, Script=Latin]
\newfontfamily\cyrillicfont{Times New Roman}[Script=Cyrillic]
\newfontfamily\cyrillicfontsf{Times New Roman}[Script=Cyrillic]
\newfontfamily\cyrillicfonttt{Courier New}[Script=Cyrillic]
\usepackage{pdfpages}
\usepackage{microtype}
\usepackage[colorlinks=false,allbordercolors={0 0 0},pdfborder={0 0 0}]{hyperref}
\setlength{\parindent}{1.5cm}
\usepackage{graphicx}
  \usepackage[
    backend=biber,
    style=numeric,
  ]{biblatex}
\addbibresource{references.bib}

\begin{document}

\begin{center}
    {\fontsize{20pt}{0pt}\selectfont 
    Национальный исследовательский университет 
    
    «Высшая школа экономики»}

    \vspace{6cm}


    {\fontsize{20pt}{0pt}\selectfont \textbf{Работа с тензорами с помощью GPGPU}}
    
    {\fontsize{16pt}{0pt}\selectfont Отчет по ИП ФКН}

    \vspace{2cm}

    {\fontsize{12pt}{0pt}\selectfont 
    Студент: К.И. Сулейманов,
    
    Руководитель: Р.А. Родригес Залепинос}

    \vspace{11cm}

    {\fontsize{18pt}{0pt}\selectfont{\centering Москва, 2025}}
    \thispagestyle{empty}
\end{center}

\newpage
\tableofcontents
\newpage

\setcounter{section}{-1}
\section{Публикация}

\subsection*{0.1 Опыт участия в конференции}
В сентябре 2025 года принял участие в международной конференции «Суперкомпьютерные дни в России». Представил стендовый доклад, отвечал на вопросы участников, получил содержательную обратную связь от исследовательского сообщества. Посетил основные доклады и тематические сессии, пообщался с коллегами, что помогло уточнить дальнейшие шаги в проекте.

Выступление и общение на конференции показали высокую заинтересованность аудитории в ускорении растровой алгебры. Полученный опыт публичного представления работы и обсуждения результатов оказался крайне полезным и вдохновляющим, помог расширить профессиональные контакты и подтвердил актуальность выбранного направления. Этот обмен идеями влияет на дальнейшее планирование экспериментов и приоритетов реализации.

Отдельно получил книгу «Методы Монте-Карло для параллельных вычислений» (Зорин А.В., Федоткин М.А.)~\cite{zorin2013montecarlo}. В ней изложены методы статистического моделирования для параллельных архитектур: генерация независимых потоков псевдослучайных чисел и случайных векторов, приближённое вычисление интегралов высокой размерности, численное решение дифференциальных уравнений и имитационное моделирование. Издание ориентировано на студентов и исследователей, работающих с численным моделированием и параллельным программированием; использую её как справочник при подготовке экспериментов.

\subsection*{0.2 Публикация}
Сулейманов К. И., Родригес Залепинос Р. А. \textbf{Обработка растровых данных на GPU в геоинформационных системах} // \textit{Суперкомпьютерные дни в России: Труды международной конференции}. 29--30 сентября 2025, Москва / Под ред. Вл. В. Воеводина. Москва: МАКС Пресс, 2025. С.~244--246. DOI: \href{https://doi.org/10.29003/m4750.978-5-317-07451-7}{10.29003/m4750.978-5-317-07451-7}.

\subsection*{0.3 Материалы стендового доклада}
Ниже приведены материалы стендового доклада: аннотация и полный постер формата A1.

\includepdf[pages=-]{poster.pdf}

\includepdf[pages=-]{A1_Poster_GPU_GIS copy.pdf}

\section{Введение}

Современные вычислительные задачи — от обучения больших языковых моделей до постобработки спутниковых снимков всё чаще опираются на операции с многомерными тензорами \cite{brown2020language,zhu2020deep}. Согласно исследованиям, рост размерности и объёма данных приводит к экспоненциальному увеличению вычислительной сложности и объёма передаваемой информации \cite{jouppi2020domain}, что делает использование GPGPU-ускорителей фактически обязательным для достижения необходимой производительности и энергоэффективности \cite{kurth2024,cudnn2014}.

Бывший научный сотрудник Стэнфордской лаборатории компьютерной графики и нынешний главный инженер по мобильным и облачным вычислениям в компании Nvidia говорит о том, что неспециализированные вычисления на графических процессорах идеально подходят для обработки данных и показывают многократное преимущество перед вычислениями на центральных процессорах, за счет возможности распараллеливания вычислений на большом объеме памяти. GPU выигрывает, когда все элементы обрабатываются одним и тем же ядром (kernel) без зависимостей~\cite{houston2007gpgpu}.

Программирование на GPGPU и использование их для массивно-параллельных вычислений становится чрезвычайно популярным. 
Во многих самых производительных суперкомпьютерах мира установлены GPGPU: \url{https://top500.org/}.
Графические ускорители также установлены в суперкомпьютере "cHARISMa" НИУ ВШЭ~\cite{charisma}.

Тензор - является мощной абстракцией, многие типы данных являются тензорами либо тензор является естественным представлением для этих данных, рис.~\ref{fig:tensors}.
Под тензором в работе понимается многомерный массив~\cite{oseledets2011tensor}.

\begin{figure}[h]
  \centering
      \centering
      \includegraphics[width=\textwidth]{tensorex.jpg}
      \caption{Типы данных, которые естественным образом представляются в виде тензоров~\cite{tutorialModels,DBLP:journals/pvldb/Zalipynis21}}
      \label{fig:tensors}
\end{figure}

Эти факты свидетельствуют о чрезвычайной важности  научно- исследовательской темы ''Работа с тензорами с помощью GPGPU''.

\section{Данные}

Растровая модель данных – это широко распространенный метод хранения
географических данных. Чаще всего эта модель представляет собой структуру, напоминающую
сетку, в которой хранятся значения с регулярными интервалами по всей площади растра.
Растры особенно хорошо подходят для хранения непрерывных данных, таких как температура
и высота над уровнем моря, но также могут содержать дискретные и категориальные данные,
например, данные о землепользовании. 

Разрешение растра задается в линейных единицах
(например, метрах) или угловых единицах (например, одной угловой секунде) и определяет
протяженность вдоль одной стороны ячейки сетки. Растры высокого (или низкого) разрешения
имеют сравнительно меньшее расстояние между ячейками сетки и больше ячеек, чем растры
низкого (или высокого) разрешения, и требуют относительно большего объема памяти для
хранения. 

Активные исследования в этой области направлены на улучшение схем сжатия и
реализацию альтернативных форм ячеек (например, шестиугольников), а также на улучшение
поддержки функций хранения и анализа растров с несколькими разрешениями ~\cite{DBLP:journals/pvldb/Zalipynis21}.

Например, Sentinel — это спутниковые данные, предоставляемые программой
Copernicus Европейского космического агентства ~\cite{sentinelData}. Они включают многоспектральные
снимки, радарные данные и данные о высотах, пригодные для мониторинга земной
поверхности, сельского хозяйства, водных ресурсов и экологии ~\cite{ArcGISBook}. 

Исследование \cite{wong2019urbanveg} показывает применение данных Sentinel: авторы оценили долю городской
растительности в дельте реки Чжуцзян (Китай) с помощью данных Sentinel-2 (NDVI,
разрешение 10 м) и валидации по высокодетальным изображениям Google. Результаты
показали корреляцию 0.97 между оценками Sentinel-2 и эталонными данными, а также выявили
значительные расхождения с методом WUDAPT level-0 (100 м): в плотных урбанизированных
зонах WUDAPT завышал городскую фракцию, а в промзонах — занижал. Подход обеспечивает
детализацию, критичную для климатического моделирования и анализа городской среды.

Другими популярными данными являются данные программы Landsat~\ref{fig:qgisLandsat}. Программа Landsat - самая длительная программа по наблюдению за Землей из космоса~\cite{landsatprogram}.

\begin{figure}[h]
  \centering
      \centering
      \includegraphics[width=\textwidth]{qgis_landsat.jpg}
      \caption{Экранный снимок QGIS: Москва, ближний инфракрасный канал спутниковой сцены Landsat}
      \label{fig:qgisLandsat}
\end{figure}

Проблема обработки растровых данных заключается в их огромных объёмах. При
увеличении разрешения растрового изображения (т.е. уменьшении размера пиксела) объём
хранимых данных резко растёт – например, при двукратном уменьшении размера ячейки объём
данных может увеличиться вчетверо, и скорость их обработки сильно снижается.

С каждым
годом появляются новые спутниковые миссии и БПЛА, генерирующие всё больше данных
высокого разрешения, поэтому проблема быстрой обработки больших растров становится всё
более актуальной и работа с растровыми данными выходит на первый план ~\cite{ArcGISBook}.


Для ускорения обработки растровых данных применяются графические процессоры (GPU)
– специальные видеокарты, способные выполнять вычисления параллельно. GPU изначально
предназначены для ускорения работы с графикой, но в последнее время их вычислительные
мощности активно используется для общих вычислительных задач. 

В инструментах
геопространственного анализа задача обработки растров выполняется GPU вместо
центрального процессора: задача дробится на множество мелких подзадач, которые GPU
выполняет параллельно с высокой скоростью, затем результаты собираются воедино \cite{kirby2013parallelizing, fuerst2016gisbased}.

В дальнейшем мы концентрируется именно на спутниковых данных, поскольку это одни из самых объемных тензорных данных в открытом доступе с большим перечнем практических задач, в которых они активно применяются.

\section{Методы}

\subsection{Растровая алгебра}


Растровая алгебра представляет собой набор методов и операций, применяемых для выполнения пространственных и математических расчётов с растровыми данными. Основная идея заключается в том, что каждое вычисление применяется одновременно ко всем ячейкам растра или группе растров, образуя новые растровые слои, содержащие результат этих операций. Наиболее распространённые операции включают арифметические, логические, статистические и тригонометрические вычисления, а также специализированные операции, такие как расчёт индексов вегетации (например, NDVI), уклона и экспозиции.

Формальные основы растровой алгебры были заложены в работах Даны Томлина, предложившего концепцию операций над растровыми слоями (алгебра карт) \cite{tomlin2012gis}. Данные подходы позволяют представить сложные пространственные задачи в виде комбинаций простых элементарных операций \cite{tomlin1994map}. В последующие годы идеи Томлина были расширены и дополнены, например, в работе Камары и соавторов \cite{camara2005towards}, предложивших включить в растровую алгебру пространственные предикаты и отношения, а также в трудах Джереми Менниса, расширившего применение растровой алгебры на многомерные и временные данные \cite{mennis2010multidimensional}.

В рамках настоящей работы основное внимание уделяется применению растровой алгебры для анализа спутниковых изображений. Одной из ключевых операций, которые планируется реализовать и ускорить с помощью GPU, является вычисление нормализованного разностного индекса вегетации (NDVI). NDVI рассчитывается по формуле:

\begin{equation}
    NDVI = \frac{NIR - RED}{NIR + RED + 1},
\end{equation}

где \(NIR\) — значение пикселя в ближнем инфракрасном канале, а \(RED\) — в красном канале изображения. Этот индекс широко используется для мониторинга состояния растительности и сельскохозяйственных угодий \cite{gebbert2019grass}.

Другим важным примером применения растровой алгебры является операция ресемплинга, предполагающая изменение пространственного разрешения изображения. Ресемплинг становится критически важным при интеграции данных с разных спутников, имеющих различные пространственные разрешения \cite{gebbert2019grass,yang2018updating}.

\begin{figure}[t]
      \centering
      \frame{\includegraphics[width=\textwidth]{Map-algebra-classes-and-their-algorithmic-patterns_W640.jpg}}
      \caption{Типы операций в растровой алгебре~\cite{carabano2018compiler}}
      \label{fig:mapalgebra}
\end{figure}

Для реализации операций растровой алгебры на GPU планируется использовать технологию CUDA, которая позволит распределить обработку растровых данных по тысячам параллельных потоков. Согласно классификации из работы \cite{carabano2018compiler}, операции растровой алгебры хорошо подходят для GPU-ускорения благодаря их массово-параллельной природе и отсутствию зависимостей между вычислениями отдельных пикселей (см. рис.~\ref{fig:mapalgebra}).

Существуют примеры успешного GPU-ускорения растровых операций, такие как реализация пространственных функций в GRASS GIS, показавшие ускорение расчётов на порядок \cite{steinbach2012accelerating}, а также ускорение интерполяции и анализа видимости на GPU, продемонстрировавшее значительный прирост производительности по сравнению с традиционными CPU-реализациями \cite{xia2011gpu}.

Таким образом, успешная реализация растровой алгебры на GPU позволит значительно сократить время обработки спутниковых данных, повысив эффективность анализа и обеспечив возможность работы с крупномасштабными геопространственными проектами.

\subsection{Wavelet}

Современные задачи обработки изображений требуют эффективных вычислений, обладающих как высокой скоростью, так и масштабируемостью. Одним из перспективных направлений является использование вей-\ влет-преобразования, позволяющего работать с изображениями в частотно-пространственном представлении. 

Данный подход обладает рядом преимуществ: он обеспечивает компактное (разреженное) представление данных, позволяет выполнять операции с прогрессивным уточнением (от грубого к детальному уровню), а также локализует действия как в пространственной, так и в частотной области. Ввиду широкой применимости и совместимости с современными стандартами (например, JPEG-2000), вейвлет-домен представляет интерес как в академических исследованиях, так и в практических ГИС-задачах.

В работе Iddo Drori и Dani Lischinski \cite{drori2003wavelet} предложен единый подход к выполнению ряда операций — таких как свёртка, 3D-вейвлет-войпинг (wavelet warping), и смешивание изображений — непосредственно в вейвлет-домене. Основной идеей является представление изображения и сопутствующих операторов в виде линейных комбинаций матриц, над которыми затем выполняется вейвлет-преобразование. Итоговая операция осуществляется над малым числом ненулевых коэффициентов, а восстановление результата производится посредством обратного преобразования.

Среди ключевых преимуществ подхода — возможность выполнения операций на различных разрешениях (multi-resolution), поддержка прогрессивной реконструкции и значительное ускорение расчётов за счёт разреженности представления. Например, при применении к задаче 3D-вейвлет-войпинга авторы показали, что в сферических проекциях алгоритм вейвлет-войпинга работает до 1.8 раза быстрее, чем классические методы. При этом сохраняется возможность интерактивной визуализации изображений в режиме реального времени благодаря поэтапной реконструкции.

Кроме того, вейвлет-преобразование позволяет реализовать как loss-\ less, так и lossy-сжатие изображений. Как показано в учебных материалах по сжатию на базе преобразования Хаара \cite{haarcompression}, детализация изображения (так называемые detail coefficients) зачастую близка к нулю. Это делает возможным удаление малых коэффициентов без существенной потери качества, что, в свою очередь, позволяет достичь высокой степени сжатия.

Для реализации вейвлет-преобразований были рассмотрены несколько базисов, включая S-преобразование (целочисленная версия Хаара) и I(2,2)-базис (интерполяционное биортогональное преобразование). Первый обеспечивает максимальную скорость, второй — наилучшее качество визуального результата при потере части информации. Выбор подходящего вейвлет-базиса определяется компромиссом между разреженностью, временем реконструкции и устойчивостью к визуальным искажениям.

Таким образом, подход к обработке изображений в вейвлет-домене представляет собой универсальное и эффективное решение, применимое как к операциям над отдельными изображениями, так и к последовательностям кадров, особенно в условиях ограниченных вычислительных ресурсов.

\section{Технологии}

Существующие GPU-библиотеки и утилиты для обработки растров: cuCIM от NVIDIA –
это библиотека для GPU-ускоренной загрузки и обработки больших файлов TIFF [7]. OpenCV
имеет модуль CUDA с GPU-версиями фильтров и преобразований [8], библиотека CuPy в
Python предоставляет NumPy – подобные массивы для вычислений на GPU. 

Библиотека
Rasterio (на основе GDAL) непосредственно CUDA не использует, но с помощью Dask/CuPy
можно распараллеливать обработку больших растров. Несмотря на наличие всех этих
инструментов, они рассчитаны на отдельные задачи и не приспособлены для массивно-
параллельной пакетной обработки геопривязанных растров в ГИС-сценариях. Аналогов
полностью предполагающих решение для массивно-параллельной обработки данных ГИС нет.

\subsection{CUDA C}

\begin{figure}[t]
      \centering
      \frame{\includegraphics[width=\textwidth]{scheme.png}}
      \caption{Модель программы из исследования \cite{kirby2013cuda}}
      \label{fig:scheme}
\end{figure}

CUDA C — это расширение языка программирования C/C++, разработанное компанией NVIDIA для реализации параллельных вычислений на графических процессорах (GPU). Основное преимущество данной технологии заключается в возможности эффективного распределения вычислений между тысячами потоков, что делает её особенно привлекательной для задач, требующих обработки больших объёмов данных \cite{nvidia2025cuda}. 

В контексте геоинформационных систем (ГИС), где распространены ресурсоёмкие операции с растровыми изображениями, использование CUDA C позволяет существенно повысить производительность по сравнению с традиционными подходами на центральном процессоре (CPU). Ввиду актуальности ускорения пространственных расчётов в рамках анализа изображений и моделирования, данная технология была выбрана для подробного рассмотрения.

В работе \cite{kirby2013cuda} была реализована параллельная обработка растровых данных в геоинформационных системах (ГИС) с использованием технологии CUDA C. Полученные результаты демонстрируют, что применение графических процессоров (GPU) позволяет достичь существенного ускорения — в среднем в 7 раз быстрее по сравнению с традиционной реализацией на платформе ArcGIS. 

Проведённые тесты показали линейную зависимость времени выполнения CUDA-функций от объёма входных данных. В частности, обработка растрового файла объёмом 1{,}96~ГБ заняла порядка 2000 секунд, в то время как обработка 93~ГБ (25 миллиардов пикселей) потребовала менее 3000 секунд при предварительно спрогнозированном времени в 3100 секунд, что подтверждает стабильность и масштабируемость подхода.

Наибольшее ускорение наблюдалось при выполнении нескольких пространственных функций на одном и том же наборе данных — прирост производительности по сравнению с ArcGIS варьировался от 3 до 22 раз. Среди исследованных функций — уклон, экспозиция, среднее значение, минимум и максимум, каждая из которых является «параллелизуемой» и хорошо масштабируется на GPU. Обработка данных осуществлялась поэтапно: чтение и разбиение на блоки, копирование в память GPU, выполнение вычислений в потоках и блоках, возвращение результатов в память CPU и последующая запись в выходной файл. 

Было выявлено, что основным узким местом остаётся фаза чтения данных с диска и загрузки их в оперативную память. Однако авторы отмечают, что даже без полной оптимизации этого этапа, внедрение CUDA обеспечивает значительное общее сокращение времени выполнения. Рекомендуемое направление дальнейших исследований — параллельная загрузка данных в оперативную память посредством многопоточности на CPU во время выполнения вычислений на GPU. Этот подход может обеспечить дополнительный прирост производительности. Кроме того, авторы предлагают применять аналогичные GPU-ориентированные методы в смежных областях, таких как дистанционное зондирование, экологический мониторинг и моделирование природных ресурсов, где также широко применяются повторяющиеся пространственные расчёты.

\subsection{QGIS на Python и PyCUDA}

\begin{figure}[t]
      \centering
      \frame{\includegraphics[width=\textwidth]{model.png}}
      \caption{Модель программы из исследования \cite{fuerst2016pycuda}}
      \label{fig:scheme}
\end{figure}

Одним из актуальных направлений в области ускорения ГИС-вычис-\ лений является использование открытых решений, интегрированных в популярные ГИС-платформы. В рамках данного обзора было решено рассмотреть реализацию анализа рельефа в среде QGIS с использованием языка Python и библиотеки PyCUDA. Выбор обусловлен тем, что PyCUDA позволяет комбинировать удобство разработки на Python с высокой производительностью вычислений на GPU, а также предоставляет доступ к параллельным вычислениям пользователям, не обладающим глубокими знаниями в области низкоуровневого программирования.

В работе \cite{fuerst2016pycuda} представлен открытый плагин для QGIS, реализующий ускорение анализа рельефа с помощью графического процессора. Плагин был написан на языке Python с использованием библиотеки PyCUDA и предназначен для выполнения вычислений уклона, экспозиции и затенения рельефа на растровых данных. Алгоритмы плагина рассчитаны на работу с окрестностью $3 \times 3$ вокруг каждого пикселя и позволяют легко модифицировать код под другие функции анализа.

Архитектура решения включает три независимых процесса: загрузку данных, их обработку на GPU и последующую запись на диск. Такая модульная структура обеспечивает параллелизм как на уровне CPU (ввод-вывод), так и на уровне GPU (вычисления), что приводит к сокращению общего времени выполнения.

Результаты экспериментов показали, что реализация на PyCUDA о-\ беспечивает 3-кратное ускорение по сравнению со встроенной реализацией QGIS на C++. Например, для растрового файла объёмом 1{,}5~ГБ общее время выполнения составило 3 минуты 35 секунд против 11 минут в QGIS. При этом само время вычислений на GPU заняло менее 2 секунд, а остальные затраты связаны в основном с операциями ввода-вывода. 

Для большего файла объёмом 12~ГБ ускорение оказалось менее выраженным (28 минут против 45 минут в QGIS), что обусловлено ограничениями по скорости чтения и записи данных с диска. Авторы отмечают, что использование SSD и чтение крупных блоков данных снижает влияние узкого места, но полностью не устраняет его.

В целом, работа демонстрирует потенциал применения GPU вычислений в составе открытых ГИС-платформ, подчёркивая значимость дальнейших исследований в области оптимизации ввода-вывода и расширения поддерживаемых алгоритмов анализа рельефа.

\section{Текущие проблемы}

Тензорные СУБД являются популярным инструментом для обработки растровых данных,
однако в них до сих пор не используются GPU для ускорения различный операций \cite{baumann2021arraydb, zalipynis2020bitfun, zalipynis2019chronosdb_action, zalipynis2018chronosdb}.

Существующие GPU-библиотеки (cuCIM, модуль CUDA в OpenCV, CuPy) ориентированы на отдельные операции и не образуют единой экосистемы для пакетной обработки геопривязанных растров; интеграция с GDAL/Rasterio или QGIS требует значительных усилий.

Выбор базиса (S-преобразование vs. I(2,2)) и порога отсечения коэффициентов влияет на качество реконструкции и время обратного преобразования; отсутствуют автоматизированные рекомендации для ГИС-практиков ~\cite{drori2003wavelet}.

\begin{figure}[t]
      \centering
      \frame{\includegraphics[width=\textwidth]{cuda-memory-hierarchy.png}}
      \caption{Иерархия памяти в CUDA \cite{beardsage2020cuda}}
      \label{fig:cuda-memory-hierarchy}
\end{figure}

И как уже было отмечено ранее, одной и самых больших проблем является то, что в современных решениях процессы чтения, вычисления и записи редко организованы как полноценный конвейер, из-за чего GPU простаивает во время операций ввода-вывода. Необходимо грамотно реализовывать работу с памятью в CUDA с пониманием её иерархии (рис. ~\ref{fig:cuda-memory-hierarchy}). 

\section{Постановка задачи}

\subsection*{Обзор литературы}

В рамках проекта необходимо изучить литературу по теме и разобраться и качественно её проанализировать. Систематизация и интеллектуальная проработка статей, их структурирование и указание связи между ними и проектом в том числе.

\subsection*{Подготовка описания постера на конференцию}

Так же был подготовлен постер по теме "Обработка растровых данных на
GPU в геоинформационных системах" под руководством научного руководителя на международную научную конференцию Суперкомпьютерные дни в России 2025. 

В ходе подготовки постера я познакомился с требованиями к обнародованию научной работы, оформлением и процессом ревью работы. Этот опыт хорошо помог мне во время подготовки данного отчета.

Необходимо подготовиться к ответу на вопросы для постер сессии во время проведения конференции. 

\subsection{Модель данных}

Для формальной постановки задачи, нам необходимо определить формальную модель данных. Будем использовать часть модели данных из~\cite{zalipynis2018chronosdb}. 

\begin{definition}[N-мерный массив]
Пусть $N\in\mathbb{N},\;N>0$.  
\emph{N-мерным массивом} (тензором) называется отображение
\[
  A : D_{1}\times D_{2}\times \dots \times D_{N} \;\longrightarrow\; T,
\]
где для каждого $i\in\{1,\dots,N\}$  
\[
  D_{i} = \{0,1,\dots,\,l_{i}-1\}\subset\mathbb{Z}, 
  \qquad l_{i}\in\mathbb{N},
\]
а $T$ — числовой тип данных (например, \texttt{int}, \texttt{float}).  

Величина $l_{i}$ называется \emph{длиной} (или \emph{размером}) $i$-й оси,  
а кортеж
\[
  \shape(A)\;:=\;(l_{1},\,l_{2},\,\dots,\,l_{N})
\]
— \emph{формой} (shape) массива (тензора) $A$.
\end{definition}

\begin{definition}[Мощность массива (тензора)]
\emph{Мощностью} (или \emph{размером}) массива $A$ называется общее число его элементов
\[
  |A|\;:=\;\prod_{i=1}^{N} l_{i}.
\]
\end{definition}

Элемент с координатами $(x_{1},\dots,x_{N})\in D_{1}\times\dots\times D_{N}$  
обозначается $A[x_{1},x_{2},\dots,x_{N}]$ и имеет тип $T$.  
Значение \texttt{NA} зарезервировано для обозначения \emph{пропущенного} элемента.

\begin{definition}[Переформатирование (reshaping) массива]\label{def:reshape}
Пусть $A:D_{1}\times\dots\times D_{N}\to T$ — массив формы $\shape(A)=(l_{1},\dots,l_{N})$. Пусть заданы новые размеры $m_{1},\dots,m_{M}$ такие, что $\prod_{i=1}^{N} l_{i}=\prod_{j=1}^{M} m_{j}$. Линейный индекс элемента исходного массива
\[
  \mathrm{lin}(x_{1},\dots,x_{N})=\sum_{k=1}^{N}x_{k}\Bigl(\prod_{j=k+1}^{N}l_{j}\Bigr)
\]
переводится в координаты выходного массива $B$ по правилу
\[
  y_{j}=\Bigl\lfloor\dfrac{\mathrm{lin}(x_{1},\dots,x_{N})}{\prod_{r=j+1}^{M}m_{r}}\Bigr\rfloor\bmod m_{j},\quad j=1,\dots,M,
\]
и значение присваивается как $B[y_{1},\dots,y_{M}]=A[x_{1},\dots,x_{N}]$. Операция обозначается $B=\operatorname{reshape}_{(m_{1},\dots,m_{M})}(A)$ и сохраняет порядок элементов в линейном представлении~\cite{zalipynis2018chronosdb}.
\end{definition}

\subsection{Обучение работе с GPU} 

Любые научно-технические работы на GPU имеют высокий порог вхождения. Поэтому для начала необходимо освоить устройство и работу GPU, научиться выполнять ввод/вывод данных и программировать GPU.  

\subsubsection{Чтение тензоров из формата GeoTIFF в GPU}

GeoTIFF сочетает привычный контейнер TIFF с полной геопривязкой, поэтому спутниковые сцены можно сразу открывать в инструментах GDAL/Rasterio. Типовой поток обработки включает чтение плитки или окна с помощью `RasterIO` (GDAL) либо cuCIM в закреплённую (pinned) память хоста и асинхронное копирование этого блока в видеопамять через `cudaMemcpyAsync`; (3) выполнение CUDA-kernel над загруженным массивом. Такой подход позволяет параллельно перекрывать ввод-вывод и вычисления \cite{gdal_gtiff}.

\subsubsection{Растровая алгебра на GPU}

Ранее уже было сказано про растровую алгебру. В этом разделе я хочу формально определить операции NDVI и ресемплинга описанные выше. Другие операции так же рассматриваются в проекте, эти две операции выбраны лишь для демонстрации в отчете. Определения также возьмем отсюда~\cite{zalipynis2018chronosdb}.

\begin{definition}[Нормализованный относительный индекс растительности, NDVI]
Пусть $A^{\text{nir}}, A^{\text{red}} : D_{1}\times D_{2}\longrightarrow \mathbb{R}_{\ge 0}\cup\{\text{NA}\}$ — взаимно геореференцированные массивы,
представляющие значения ближнего инфракрасного ($\text{NIR}$) и красного ($\text{Red}$) диапазонов
спутникового снимка соответственно.  
Для каждой ячейки $(x,y)\in D_{1}\times D_{2}$ \emph{NDVI} определяется как
\[
  \operatorname{NDVI}(x,y) \;=\;
      \dfrac{A^{\text{nir}}[x,y]\;-\;A^{\text{red}}[x,y]}
            {A^{\text{nir}}[x,y]\;+\;A^{\text{red}}[x,y]\;+\;1}
\]
Значение $\operatorname{NDVI}(x,y)\in[-1,1]$ служит прокси-оценкой
фотосинтетической активности растительного покрова.
\end{definition}

\begin{figure}[t]
      \centering
      \frame{\includegraphics[width=\textwidth]{Terrain-Forest-Dataset-Showing-a-Surface-Level-of-Detail-for-Different-Pyramid-Levels.png}}
      \caption{Визуализация ресемплирования \cite{abdallah2010terrainforest}}
      \label{fig:resampling}
\end{figure}

\begin{definition}[k-кратное ресемплирование усреднением]\label{def:resample}
Пусть $A : D_{1}\times D_{2}\longrightarrow T$ — двумерный массив,
где $D_{i}=\{0,1,\dots,l_{i}-1\}$ и $k\in\mathbb{N},\,k\ge 2$ — коэффициент
декрета (обычно $k=2$).  
Определим уплотнённый массив
\[
  R_{k}(A) : D_{1}^{\prime}\times D_{2}^{\prime}\longrightarrow T,
  \qquad
  D_{i}^{\prime}=\Bigl\{0,1,\dots,\Bigl\lfloor\frac{l_{i}}{k}\Bigr\rfloor-1\Bigr\},
\]
по правилу
\[
\begin{aligned}
  R_{k}(A)[u,v]
    &= \avg\!\Bigl(\bigl\{\,A[x,y] \;\big|\;
         uk \le x < (u+1)k,\;
         vk \le y < (v+1)k,\;
         A[x,y]\neq\text{NA}\bigr\}\Bigr),\\[0.5em]
    &\quad (u,v)\in D_{1}^{\prime}\times D_{2}^{\prime}.
\end{aligned}
\]
где $\avg(\varnothing)=\text{NA}$.  
Таким образом, каждая ячейка выходного массива есть среднее
ненулевых значений соответствующего блока $k\times k$ входного массива.
При $k=2$ $R_{k}(A)$ соответствует одному уровню
\emph{многоуровневой пирамиды разрешений} (рис.~\ref{fig:resampling}).
\end{definition}

\subsection{Сравнение скорости GPU и CPU}

Важной частью является сравнение реализованной технологии с \\ gdal\_calc — это консольная утилита пакета GDAL, которая выполняет арифметические и логические операции над растровыми наборами данных, интерпретируя выражения в синтаксисе NumPy. Она позволяет автоматизировать комплексную обработку растров (например, вычисление индексов, маскирование, объединение каналов) с сохранением пространственной привязки и метаданных исходных файлов \cite{gdal_calc}.

Это неоходимо для того, чтобы подвести итоги проделанной работы и проанализировать полученные результаты ускорения.

\subsection{Ускорение вычислений с помощью вейвлетов}

Про данный метод также было написано ранее. Отмечу лишь, что этот метод достаточно продвинутым и обладает масштабируемостью. Так что её изучение и реализация считаю крайне важным пунктом работы.

\subsection{Реализация reshaping}
На основе формализации операции reshaping (см. определение~\ref{def:reshape} и \cite{zalipynis2018chronosdb}) запланированы два этапа реализации:
\begin{itemize}
  \item многопоточная CPU-версия с сохранением порядка элементов в линейной нумерации и поддержкой форматов NetCDF/GeoTIFF (базовые данные: NOAA NCEP Reanalysis, примеры из ArcGIS и Panoply);
  \item порт на GPU (CUDA) с сопоставлением потоков линейным индексам и сравнением скорости с CPU, включая профилирование копирования и вычислений.
\end{itemize}

\section{Текущие наработки}

\subsubsection*{Выполнено}
\begin{itemize}
    \item Проведено изучение и качественный обзор литературы, написан отчет по результатам всей проделанной работы, чем является данный документ.
    \item Подготовлен постер на международную научную конференцию Суперкомпьютерные дни в России 2025 непосредственно по теме исследовательского проекта.
    \item Получен доступ к Яндекс.Облаку с графическими картами NVIDIA Tesla V100 содержащими 5120 ядер CUDA, позволяющих выполнять высокопроизводительные вычисления (High Performance Computing, HPC).
    \item Подготовленный и загружены на сервер данные Landsat — это откалиброванные мультиспектральные сцены со спутников Landsat-1…-9: 7–11 полос (синяя, зелёная, красная, NIR, две SWIR, два TIRS); пространственная дискретизация — 15 м (панхром), 30 м (мультиспектр) и 100 м (TIRS, ресэмплируется до 30 м); размер сцены ≈ 185 × 180 км; повторное покрытие — 16 суток (8 суток при парной съёмке); доступны форматы Collection 2 Level-1 (L1TP/L1GT/L1GS), Level-2 (Surface Reflectance, Surface Temperature), Analysis-Ready Data (ARD) и Harmonized Landsat–Sentinel-2 (HLS) с метаданными MTL и масками QA.
\end{itemize}

\subsubsection*{В процессе}
Реализация методов растровой алгебры на подготовленных данных с освоением устройства работы с CUDA и памятью GPU.

\section{Перспективы на будущее}
После реализации методов растровой алгебры планируется реализовать вейвлет-преобразования из статей \cite{drori2003wavelet, haarcompression} на данных описанных выше. Затем будут изучаться альтернативные методы, алгоритмы и технологии применимые в проекте. Будет исследован потенциал использования вейлет-преобразований для ускорения растровой алгебры и других операций над тензорами.

В сентябре планируется выступить с постером на конференции Суперкомпьютерные дни в России и получить опыт и знания от участия в мероприятиях такого формата. 

\newpage
\section*{\textcolor{red}{$\bigstar$} Постер на международной научной конференции Суперкомпьютерные дни в России 2025}

По результатам экспертизы (рецензирования), описание постера принято на международную научную конференцию Суперкомпьютерные дни в России 2025: \url{https://russianscdays.org/}

Описание постера будет опубликовано в электронном виде на сайте конференции по завершению работы конференции и направлено на индексацию в \textbf{РИНЦ (Российский индекс научного цитирования)}. Основные труды конференция публикуются в Springer.

Суперкомпьютерные дни в России -- ведущая конференция в России по моей теме (''работа с тензорами с помощью GPGPU''). Представление постера на этой конференции и участие в ней поможет мне получить большой опыт участия в крупных научных мероприятиях и обратную связь от экспертов в этой области.

Описание постера содержит информацию о планах и начальных результатах недавно начатого научного исследования, рис.~\ref{fig:poster}.

\begin{figure}[t]
      \centering
      \frame{\includegraphics[width=\textwidth]{poster_accept.png}}
      \caption{Оповещение о принятии постера на конференцию ''Суперкомпьютерные дни в России 2025''}
      \label{fig:poster}
\end{figure}

\newpage
\clearpage 
\printbibliography

    

\end{document}
